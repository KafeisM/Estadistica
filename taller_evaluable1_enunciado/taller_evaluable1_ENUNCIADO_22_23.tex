% Options for packages loaded elsewhere
\PassOptionsToPackage{unicode}{hyperref}
\PassOptionsToPackage{hyphens}{url}
%
\documentclass[
]{article}
\usepackage{amsmath,amssymb}
\usepackage{lmodern}
\usepackage{iftex}
\ifPDFTeX
  \usepackage[T1]{fontenc}
  \usepackage[utf8]{inputenc}
  \usepackage{textcomp} % provide euro and other symbols
\else % if luatex or xetex
  \usepackage{unicode-math}
  \defaultfontfeatures{Scale=MatchLowercase}
  \defaultfontfeatures[\rmfamily]{Ligatures=TeX,Scale=1}
\fi
% Use upquote if available, for straight quotes in verbatim environments
\IfFileExists{upquote.sty}{\usepackage{upquote}}{}
\IfFileExists{microtype.sty}{% use microtype if available
  \usepackage[]{microtype}
  \UseMicrotypeSet[protrusion]{basicmath} % disable protrusion for tt fonts
}{}
\makeatletter
\@ifundefined{KOMAClassName}{% if non-KOMA class
  \IfFileExists{parskip.sty}{%
    \usepackage{parskip}
  }{% else
    \setlength{\parindent}{0pt}
    \setlength{\parskip}{6pt plus 2pt minus 1pt}}
}{% if KOMA class
  \KOMAoptions{parskip=half}}
\makeatother
\usepackage{xcolor}
\usepackage[margin=1in]{geometry}
\usepackage{color}
\usepackage{fancyvrb}
\newcommand{\VerbBar}{|}
\newcommand{\VERB}{\Verb[commandchars=\\\{\}]}
\DefineVerbatimEnvironment{Highlighting}{Verbatim}{commandchars=\\\{\}}
% Add ',fontsize=\small' for more characters per line
\usepackage{framed}
\definecolor{shadecolor}{RGB}{248,248,248}
\newenvironment{Shaded}{\begin{snugshade}}{\end{snugshade}}
\newcommand{\AlertTok}[1]{\textcolor[rgb]{0.94,0.16,0.16}{#1}}
\newcommand{\AnnotationTok}[1]{\textcolor[rgb]{0.56,0.35,0.01}{\textbf{\textit{#1}}}}
\newcommand{\AttributeTok}[1]{\textcolor[rgb]{0.77,0.63,0.00}{#1}}
\newcommand{\BaseNTok}[1]{\textcolor[rgb]{0.00,0.00,0.81}{#1}}
\newcommand{\BuiltInTok}[1]{#1}
\newcommand{\CharTok}[1]{\textcolor[rgb]{0.31,0.60,0.02}{#1}}
\newcommand{\CommentTok}[1]{\textcolor[rgb]{0.56,0.35,0.01}{\textit{#1}}}
\newcommand{\CommentVarTok}[1]{\textcolor[rgb]{0.56,0.35,0.01}{\textbf{\textit{#1}}}}
\newcommand{\ConstantTok}[1]{\textcolor[rgb]{0.00,0.00,0.00}{#1}}
\newcommand{\ControlFlowTok}[1]{\textcolor[rgb]{0.13,0.29,0.53}{\textbf{#1}}}
\newcommand{\DataTypeTok}[1]{\textcolor[rgb]{0.13,0.29,0.53}{#1}}
\newcommand{\DecValTok}[1]{\textcolor[rgb]{0.00,0.00,0.81}{#1}}
\newcommand{\DocumentationTok}[1]{\textcolor[rgb]{0.56,0.35,0.01}{\textbf{\textit{#1}}}}
\newcommand{\ErrorTok}[1]{\textcolor[rgb]{0.64,0.00,0.00}{\textbf{#1}}}
\newcommand{\ExtensionTok}[1]{#1}
\newcommand{\FloatTok}[1]{\textcolor[rgb]{0.00,0.00,0.81}{#1}}
\newcommand{\FunctionTok}[1]{\textcolor[rgb]{0.00,0.00,0.00}{#1}}
\newcommand{\ImportTok}[1]{#1}
\newcommand{\InformationTok}[1]{\textcolor[rgb]{0.56,0.35,0.01}{\textbf{\textit{#1}}}}
\newcommand{\KeywordTok}[1]{\textcolor[rgb]{0.13,0.29,0.53}{\textbf{#1}}}
\newcommand{\NormalTok}[1]{#1}
\newcommand{\OperatorTok}[1]{\textcolor[rgb]{0.81,0.36,0.00}{\textbf{#1}}}
\newcommand{\OtherTok}[1]{\textcolor[rgb]{0.56,0.35,0.01}{#1}}
\newcommand{\PreprocessorTok}[1]{\textcolor[rgb]{0.56,0.35,0.01}{\textit{#1}}}
\newcommand{\RegionMarkerTok}[1]{#1}
\newcommand{\SpecialCharTok}[1]{\textcolor[rgb]{0.00,0.00,0.00}{#1}}
\newcommand{\SpecialStringTok}[1]{\textcolor[rgb]{0.31,0.60,0.02}{#1}}
\newcommand{\StringTok}[1]{\textcolor[rgb]{0.31,0.60,0.02}{#1}}
\newcommand{\VariableTok}[1]{\textcolor[rgb]{0.00,0.00,0.00}{#1}}
\newcommand{\VerbatimStringTok}[1]{\textcolor[rgb]{0.31,0.60,0.02}{#1}}
\newcommand{\WarningTok}[1]{\textcolor[rgb]{0.56,0.35,0.01}{\textbf{\textit{#1}}}}
\usepackage{graphicx}
\makeatletter
\def\maxwidth{\ifdim\Gin@nat@width>\linewidth\linewidth\else\Gin@nat@width\fi}
\def\maxheight{\ifdim\Gin@nat@height>\textheight\textheight\else\Gin@nat@height\fi}
\makeatother
% Scale images if necessary, so that they will not overflow the page
% margins by default, and it is still possible to overwrite the defaults
% using explicit options in \includegraphics[width, height, ...]{}
\setkeys{Gin}{width=\maxwidth,height=\maxheight,keepaspectratio}
% Set default figure placement to htbp
\makeatletter
\def\fps@figure{htbp}
\makeatother
\setlength{\emergencystretch}{3em} % prevent overfull lines
\providecommand{\tightlist}{%
  \setlength{\itemsep}{0pt}\setlength{\parskip}{0pt}}
\setcounter{secnumdepth}{-\maxdimen} % remove section numbering
\ifLuaTeX
  \usepackage{selnolig}  % disable illegal ligatures
\fi
\IfFileExists{bookmark.sty}{\usepackage{bookmark}}{\usepackage{hyperref}}
\IfFileExists{xurl.sty}{\usepackage{xurl}}{} % add URL line breaks if available
\urlstyle{same} % disable monospaced font for URLs
\hypersetup{
  pdftitle={Taller Evaluable 1, FIFA 2022-23},
  hidelinks,
  pdfcreator={LaTeX via pandoc}}

\title{Taller Evaluable 1, FIFA 2022-23}
\author{}
\date{\vspace{-2.5em}}

\begin{document}
\maketitle

\begin{flushright}\includegraphics[width=0.7\linewidth]{logoM3} \end{flushright}

\hypertarget{taller-evaluable-datos-fifa-2023}{%
\section{Taller evaluable datos FIFA
2023}\label{taller-evaluable-datos-fifa-2023}}

Cada pregunta vale 1 punto. Se puntúa la presentación, la claridad y que
los dibujos están completos. Este taller está pensado para resolver con
R-base pero podéis utilizar tidyverse u otros paquetes de R.

En la web de kaggle
\href{https://www.kaggle.com/datasets/bryanb/fifa-player-stats-database}{FIFA23
OFFICIAL DATASET}. Contiene todos los data sets de de datos básicos de
FIFA17 to FIFA23 del videojuego.

Las siguientes preguntas son relativas al data set
\texttt{players\_23.csv}, que se adjunta con la práctica.

Hay que contestar con código R explicar adecuadamente cada salida. Subid
a la activad el Rmd y el html.

Rellenad estos datos:

\textbf{PONED NOMBRE DEL GRUPO}

\begin{itemize}
\tightlist
\item
  Apellidos, Nombre Alumno
\item
  Apellidos, Nombre Alumno
\item
  Apellidos, Nombre Alumno
\item
  Apellidos, Nombre Alumno
\end{itemize}

\hypertarget{carga-de-datos}{%
\subsection{Carga de datos}\label{carga-de-datos}}

Tenéis que carga los datos con el código que se ve a continuación.
Visualizar y explorara el data set y averiguar de qué tipo son cada una
de las variables y en qué tipo de fichero están guardadas. El código
carga los datos en un data frame llamado \texttt{datos} con la función
\texttt{read.csv}. Debéis entender clases y tipos de datos de cada de
cada columna de datos. El parámetro \texttt{encoding} es necesario para
cargar debidamente los acentos y caracteres especiales. Lo que obtenemos
es un data frame de 18539 observaciones (filas) y 90 variables
(columnas).

Cargaremos todas las variables de texto como factor con el parámetro
\texttt{stringsAsFactors\ =\ TRUE}

\begin{Shaded}
\begin{Highlighting}[]
\NormalTok{datos }\OtherTok{=} \FunctionTok{read.csv}\NormalTok{(}\StringTok{"players\_fifa23.csv"}\NormalTok{,}
  \AttributeTok{encoding=}\StringTok{"UTF{-}8"}\NormalTok{,}\AttributeTok{stringsAsFactors =} \ConstantTok{TRUE}\NormalTok{)}\CommentTok{\# cambia tu path}
\CommentTok{\#str(datos)}
\CommentTok{\#names(datos)}
\end{Highlighting}
\end{Shaded}

Las variables de la 1 (\texttt{NationalTeam}) a la
31(\texttt{NationalPosition}) son variables de perfil del jugador: su
nombre, su equipo su sueldo su número de camiseta\ldots{} El resto de
variables de la 34 (\texttt{pace}) a la 90 (\texttt{rb}) son variables
numéricas enteras con valores de 0 a 100 que parametrizan cómo es el
jugador el el juego FIFA player 2023.

\hypertarget{pregunta-1}{%
\subsection{Pregunta 1}\label{pregunta-1}}

Las selecciones europeas que han ganado un mundial son

\begin{Shaded}
\begin{Highlighting}[]
\NormalTok{eur}\OtherTok{=}\FunctionTok{c}\NormalTok{(}\StringTok{"England"}\NormalTok{,}\StringTok{"France"}\NormalTok{,}\StringTok{"Germany"}\NormalTok{,}\StringTok{"Italy"}\NormalTok{,}\StringTok{"Portugal"}\NormalTok{,}\StringTok{"Spain"}\NormalTok{)}
\NormalTok{eur}
\end{Highlighting}
\end{Shaded}

\begin{verbatim}
## [1] "England"  "France"   "Germany"  "Italy"    "Portugal" "Spain"
\end{verbatim}

Generar un data frame con el nombre \texttt{fifa23\_eur} con los
jugadores de estas selecciones.

\hypertarget{pregunta-2}{%
\subsection{Pregunta 2}\label{pregunta-2}}

Crea un data frame \texttt{fifa23\_eur} que contenga a TODOS los
jugadores de los clubs encontrados en el ejercicio anterior.

\hypertarget{pregunta-3}{%
\subsection{Pregunta 3}\label{pregunta-3}}

Calcular la media y la desviación típica del valor de cada selección
nacional cada equipo del data frame \texttt{fifa23\_eur}.

Calcular la media y la desviación típica EN MILES de euros del valor de
cada jugador \texttt{ValueEUR} de cada selección nacional del frame
\texttt{fifa23\_eur} por posición en el campo delantera media y defensa.

\hypertarget{pregunta-4}{%
\subsection{Pregunta 4}\label{pregunta-4}}

Discretiza la variable \texttt{ValueEUR} de \texttt{fifa23\_eur} en los
4 niveles siguientes: ``Cuartil\_1'', ``Cuartil\_2'', ``Cuartil\_3'' y
``Cuartil\_4'', según los cortes por la función \texttt{quantile}para
0.25,0.5 y 0.75. La variable resultante Value\_Level tiene que ser un
factor ordenado en orden creciente de valor.

\hypertarget{pregunta-5}{%
\subsection{Pregunta 5}\label{pregunta-5}}

¿Qué selección tiene a más jugadores en del intervalos de Valor máximo
calculado en el ejercicio anterior?

Estudiar la función \texttt{droplevels} para quitar los niveles de las
selecciones que no aparecen.

\hypertarget{pregunta-6}{%
\subsection{Pregunta 6}\label{pregunta-6}}

¿Respecto al tiro cuántos zurdos, diestros y ambidiestros (3) (buscad
qué variable es e interpretar su valor de 1 a 5 hay entre todos los
jugadores de \texttt{fifa23\_eur}? Construir una variable llamada
\texttt{foot} que tenga por niveles ``left'', ``right'',``ambidextrous''
¿Qué selección tiene mayor cantidad de zurdos (decidid que es zurdo
diestro y ambidiestro)?

\hypertarget{pregunta-7}{%
\subsection{Pregunta 7}\label{pregunta-7}}

Calcular la la tabla de contingencia (frecuencias absolutas) por
posición \texttt{NationalPosition} contra \texttt{foot}. contingencia
con las variable \texttt{foot}. Calcular la tabla global de proporciones
de \texttt{NationalPosition} y \texttt{foot}. Calcular la tabla de
proporciones marginales de \texttt{foot} por (condicionada a)
\texttt{NationalPosition}.

\hypertarget{pregunta-8}{%
\subsection{Pregunta 8}\label{pregunta-8}}

Calcular diagramas de barras adosados para la primera tabla del
ejercicio anterior y un diagrama de mosaico de la segunda tabla. Poned
una leyenda y nombre del gráfico y comentar los resultados con un
pequeño párrafo.

\hypertarget{pregunta-9}{%
\subsection{Pregunta 9}\label{pregunta-9}}

Comparar la distribución del score \texttt{Overall} con un boxplot para
las 6 selecciones. Decorar adecuadamente el resultado. Comentar los
resultados.

\hypertarget{pregunta-10}{%
\subsection{Pregunta 10}\label{pregunta-10}}

Generar un data frame \texttt{fifa23\_ame} que contenga exclusivamente a
las 6 selecciones de América que van al mundial 2022.

\begin{Shaded}
\begin{Highlighting}[]
\NormalTok{ame}\OtherTok{=}\FunctionTok{c}\NormalTok{(}\StringTok{"Argentina"}\NormalTok{,}\StringTok{"Brazil"}\NormalTok{,}\StringTok{"Canada"}\NormalTok{,}\StringTok{"Mexico"}\NormalTok{,}\StringTok{"Ecuador"}\NormalTok{,}\StringTok{"United States"}\NormalTok{ )}
\end{Highlighting}
\end{Shaded}

Generar un data frame \texttt{fifa23\_ame}. Comparar la distribución del
score \texttt{overall} para TODOS los jugadores de las 6 selecciones de
europa y TODOS los jugadores de las seis selecciones de América.
Dibujando un histograma con la función \emph{kernel} en un solo gráfico.
Comentar los resultados.

\end{document}
